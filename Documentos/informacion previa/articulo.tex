\documentclass[12pts]{article}

\begin{document}
\begin{center}
Some New results on the Levy, Levy and Solomon microscopic stock market model\\
\small{E.Zschischang, T. Lux}\\
\small{Physic A 5834, p 1-11, 2001}
\end{center}
\section*{El modelo set-up y resultados previos}
\subsection*{Modelo estructural}
\quad Al comienzo de cada periodo cada inversor $i$ necesita dividir su riqueza entera $W(t)$ en acciones y bonos. Su riqueza, por lo tanto, es divididaen una fraccion $X(i)$ que mantiene en acciones en el tiempo $t$ y una fraccion $1-X(i)$n que mantiene en bonos. Los crteditos y las ventas cortaas no estan permitidas, $X(i)$ es limitada de 0 a 1, es decir, $0.01\leq X(i)\leq 0.99$. Adicionalmente, el modelo asume que el numero de inversores, $n$, como la oferta de acciones, $N_A$ son fijas. Aparte de una funcion de utilidad identica $U(W_{t+1})$ inversores en el comienzo tambien tienen la misma cantidad de riqueza y el mismo numero de acciones. Whereas los bonos sin riesgos pagan una taza de interes dijo, $r$, el retorno de mercad $H_t$ varia con el tiempo y es sujeto a incerteza. Esta compuesto de dos componentes: primero, incluye ganancias o perdidas resutante de los cmabios del precio del mercad ($P_t$). Segundo, accionistas reciben un dividendo pagado $D_t$ por periodo que es asumido a seguir un camino de crecimiento estocastico. Retornos de las acciones con riesgo, por lo tanto, son dados por
\begin{equation}
H_t=\frac{P_t-P_{t-1}+D_t}{P_t-1}
\end{equation}
\quad En la version basica del modelo, preferencias de los inversores son descritas por una funcion de utilidad logaritmica que es acorde con la suposicion usual de disminución marginal de utilidad de riqueza y adversion de inversores. De retornos futuros aparece en el problema de maximizacion de utilidad de inversores, su esectativa sobre el precio futuro y retornos tiene que ser considerado. En Levy, Levy y Solomon, la supoicion permanente es que inversores tienen una memoria limitada de largo $k$ periodos y esperan que el retorno observado en este intervalo ocurra en el siguiente periodo con probabilidad igual a $1/k$. Dado estas expectativas, la utilidad esperada $E[U(x(I))] $ puede ser maximizado con respecto al numero de acciones demandadas por el agente. En sus simuluaciones, Levy, Levy y Solomon consideraron uno o más grupos de inversores con span memoria identica $k$. Una vez el numero  optimo de  acciones ha sido calculado para cada grupo inversor, cada demanda de inversion individuual es calculada agregando un numero aleatorio distribuido normalmente $\varepsilon_i$ to the outcome of the maximization process. Esto deja la heterogeneidad en grupos con fluctuaciones alrededor del largo promedio de la acciones demandada. Con todas las funciones de demanda individuales dadas, el nuevo precio del mercado es calculado como el precio de equilibrio en el mercado (es decir, un precio que deja identico la demanda y la oferta).
\subsection*{Resultados Previos}  
\end{document}
